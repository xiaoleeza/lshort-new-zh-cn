\chapter{排版样式设定}

\section{字体和字号}

\subsection{字体变换命令}
\index{font}\index{font size} 
\LaTeX{} 根据文档的逻辑结构(章节、脚注等)来选择合适的字体样式和字号。
在某些情况下,你可能会想要手动改变文档使用的字体样式和字号。
为了完成这个目的,你可以使用表 \ref{tab:fonts} 和表 \ref{tab:sizes} 中
列出的那些命令。每个字体的实际大小是一个设计问题,并且它依赖于文档所使用
的文档类及其选项。表 \ref{tab:ptsizes} 列出了这些命令在标准文档类中的绝对大小,单位为 pt。

\begin{example}
{\small The small and
\textbf{bold} Romans ruled}
{\Large all of great big
{\itshape Italy}.}
\end{example}

\LaTeXe 的一个重要特征是:字体的各种属性是相互独立的,这意味着你可以改变字体的大小,而仍然保留字体原有的粗体或者斜体的特性。

\LaTeX\ 提供了两组修改字体的命令,见表 \ref{tab:fonts}。其中诸如 \cmd{bfseries} 形式的命令将会影响之后所有的字符,
如果想要让它在局部生效,需要用花括号建立\textbf{分组},也就是写成 \marg*{\cmd{bfseries}\ \Arg{some text}} 这样的形式;
对应的 \cmd{textbf} 形式带一个参数,它只会改变参数内部的字体,相比之下更为常用。

在公式中,直接使用 \cmd{textbf} 等命令不会起效,甚至报错。如果你需要在公式中修改字体,
你需要 \LaTeX\ 为此提供的专门命令,详见表 \ref{tab:mathfonts}。

\begin{table}[!bp]
\caption{字体命令。} \label{tab:fonts}
\centering
\begin{tabular}{@{}rr@{\quad}l@{}}
\hline
\fni{rmfamily} & \fni{textrm}\marg*{\ldots}   & \textrm{\wi{roman}}      \\
\fni{sffamily} & \fni{textsf}\marg*{\ldots}   & \textsf{\wi{sans serif}} \\
\fni{ttfamily} & \fni{texttt}\marg*{\ldots}   & \texttt{\wi{typewriter}} \\[\medskipamount]
\fni{mdseries} & \fni{textmd}\marg*{\ldots}   & \textrm{medium}          \\
\fni{bfseries} & \fni{textbf}\marg*{\ldots}   & \textsf{\wi{bold face}}  \\[\medskipamount]
\fni{upshape}  & \fni{textup}\marg*{\ldots}   & \textrm{\wi{upright}}    \\
\fni{itshape}  & \fni{textit}\marg*{\ldots}   & \textrm{\wi{italic}}     \\
\fni{slshape}  & \fni{textsl}\marg*{\ldots}   & \textrm{\wi{slanted}}    \\
\fni{scshape}  & \fni{textsc}\marg*{\ldots}   & \textrm{\wi{Small Caps}} \\[\medskipamount]
\fni{em}       & \ci{emph}\marg*{\ldots}      & \emph{\wi{emphasized}}   \\
\fni{normalfont}  & \fni{textnormal}\marg*{\ldots}   & \textnormal{\wi{normal font}} \\
\hline
\end{tabular}
\end{table}

\index{font size}
\begin{table}[!tbp]
\centering
\caption{字号。} \label{tab:sizes}
\begin{tabular}{@{}ll}
\hline
\fni{tiny}      & \tiny        tiny font \\
\fni{scriptsize}   & \scriptsize  very small font\\
\fni{footnotesize} & \footnotesize  quite small font \\
\fni{small}        &  \small            small font \\
\fni{normalsize}   &  \normalsize  normal font \\
\fni{large}        &  \large       large font \\
\hline
\end{tabular}%
\qquad\begin{tabular}{ll@{}}
\hline
\fni{Large}        &  \Large       larger font \\[5pt]
\fni{LARGE}        &  \LARGE       very large font \\[5pt]
\fni{huge}         &  \huge        huge \\[5pt]
\fni{Huge}         &  \Huge        largest \\
\hline
\end{tabular}
\end{table}

\begin{table}[!tbp]
\centering
\caption{标准文档类中的字号大小。}\label{tab:ptsizes}
\begin{tabular}{lrrr}
\hline
字号 & 10pt 选项(默认)& 11pt 选项 & 12pt 选项 \\
\cmd{tiny}       & 5pt  & 6pt & 6pt\\
\cmd{scriptsize} & 7pt  & 8pt & 8pt\\
\cmd{footnotesize} & 8pt & 9pt & 10pt \\
\cmd{small}        & 9pt & 10pt & 11pt \\
\cmd{normalsize} & 10pt & 11pt & 12pt \\
\cmd{large}      & 12pt & 12pt & 14pt \\
\cmd{Large}      & 14pt & 14pt & 17pt \\
\cmd{LARGE}      & 17pt & 17pt & 20pt\\
\cmd{huge}       & 20pt & 20pt & 25pt\\
\cmd{Huge}       & 25pt & 25pt & 25pt\\
\hline
\end{tabular}
\end{table}

\begin{table}[!tbp]
\begin{minipage}{\linewidth}
\centering
\caption{数学字体\protect\footnote{\cmd{mathcal} 只有大写字母部分。小写字母部分被符号占据。}} \label{tab:mathfonts}
\begin{tabular}{@{}ll@{}}
\fni{mathrm}\marg*{\ldots}&     $\mathrm{Roman\ Font}$\\
\fni{mathbf}\marg*{\ldots}&     $\mathbf{Boldface\ Font}$\\
\fni{mathsf}\marg*{\ldots}&     $\mathsf{Sans\ Serif\ Font}$\\
\fni{mathtt}\marg*{\ldots}&     $\mathtt{Typewriter\ Font}$\\
\fni{mathit}\marg*{\ldots}&     $\mathit{Italic\ Font}$\\
\fni{mathcal}\marg*{\ldots}&    $\mathcal{CALLIGRAPHIC\ FONT}$\\
\fni{mathnormal}\marg*{\ldots}& $\mathnormal{Normal\ Font}$\\
\end{tabular}
\end{minipage}
\end{table}

使用字号命令的时候,通常也需要用花括号进行分组,如同 \cmd{rmfamily} 那样。

\begin{example}
He likes {\LARGE large and
{\small small} letters}.
\end{example}

字号命令本身还会改变行距,但行距的改变只有在字号命令的作用区域内\textbf{分段}时才生效。
因此,用于分组的后括号 \texttt\} 不应该太早出现。注意下面两个例子中\ci{par} 命令的位置%
\footnote{\cmd{par} 相当于一个空行。}。

\begin{example}
{\Large Don't read this!
 It is not true.
 You can believe me!\par}
\end{example}

\begin{example}
{\Large This is not true either.
But remember I am a liar.}\par
\end{example}

\subsection{选用字体宏包}

尽管到了这里你知道了如何切换粗体、斜体等等,以及如何改变字号,
但你依然用着 \LaTeX\ 默认的那套、由 \TeX\ 程序的开发者高德纳亲自制作的 Computer Modern 字体。
有的人可能很喜欢 Times 或者 Palatino,或者更好看的字体。这些字体样式的自由设置在 \LaTeX\ 里还不太容易。

幸好大部分时候,许多字体宏包为我们完成了整套配置,我们可以在调用宏包之后,照常使用 \cmd{bfseries} 或 \cmd{ttfamily} 等我们熟悉的命令。
表 \ref{tab:font-pkgs} 列出了较为常用的字体宏包,其中相当多的字体包还配置了数学字体,或者文本、数学字体兼而有之。
更多的的字体配置参考 \cite{survey,fontcatalogue}。

\begin{table}[!p]
\centering
\caption{常见的 \LaTeX\ 字体宏包。}\label{tab:font-pkgs}
\begin{tabular*}{\linewidth}{@{\extracolsep{0pt plus 1fill}}cp{0.65\linewidth}@{}}
 \hline
 \pai{lmodern}     & Latin Modern 字体,对 Computer Modern 字体的扩展  \\
 \pai{txfonts}     & Palatino 风格的字体宏包  \\
 \pai{pxfonts}     & Times 风格的字体宏包  \\
 \pai{newtxtext},\pai{newtxmath}  & \pkg{txfonts} 的改进版本,分别设置文本和数学字体  \\
 \pai{newpxtext},\pai{newpxmath}  & \pkg{pxfonts} 的改进版本,分别设置文本和数学字体  \\
 \pai{mathptmx}    & \pkg{psnfss} 组件之一,Times 风格,较为陈旧,不推荐使用  \\
 \pai{mathpazo}    & \pkg{psnfss} 组件之一,Palatino 风格,较为陈旧,不推荐使用  \\
 \pai{ccfonts}     & Concrete 风格字体 \\
 \pai{euler}       & Euler 风格数学字体,也出自于高德纳之手 \\
 \pai{fourier}     & fourier 风格数学字体 \\
 \pai{arev}        & Arev 风格的无衬线字体 \\
 \pai{cmbright}    & 仿 Computer Modern 风格的无衬线字体 \\
 \pai{libertine}   & Linux Libertine 衬线字体 \\
 \pai{droid}       & Droid Serif/Droid Sans 等 \\
 \pai{inconsolata} & Inconsolata 一款不错的开源等宽字体 \\
 \hline
\end{tabular*}
\end{table}

\subsection{\hologo{XeLaTeX} 下使用 \pkg{fontspec} 宏包更改字体}
\paih{fontspec}

\hologo{XeLaTeX} 能够支持直接调用系统安装的 \texttt{.ttf} 或 \texttt{.otf} 格式字体%
\footnote{Linux 下的 \TeX\ Live 为了支持系统安装的字体,需要额外的配置。详见附录 \ref{app:install}。}。相比于上一小节,我们有了更多修改字体的余地。

\hologo{XeLaTeX} 下支持用户调用字体的宏包是 \pkg{fontspec}。宏包提供了几个设置全局字体的命令,设置 \cmd{rmfamily} 等对应命令的默认字体:
\begin{command}
\cmd{setmainfont}\oarg{font features}\marg{font name} \\
\cmd{setsansfont}\oarg{font features}\marg{font name} \\
\cmd{setmonofont}\oarg{font features}\marg{font name}
\end{command}
\cih{setmainfont}
\cih{setsansfont}
\cih{setmonofont}
其中 \marg{font name} 使用字体的文件名(带扩展名)或者字体的英文名称。\Arg{font features} 用来手动配置对应的粗体或斜体
(一般情况下是自动配置的),如 \texttt{Bold\-Font=\Arg{font name},Italic\-Font=\Arg{font name}}。
\Arg{font features} 还能配置字体本身的各种特性,这里不再赘述,感兴趣的读者请参考 \pkg{fontspec} 宏包的帮助文档。

\subsection{使用 \pkg{xeCJK} 宏包更改中文字体}
\paih{xeCJK}

前文已经介绍过的 \pkg{xeCJK} 宏包使用了和 \pkg{fontspec} 宏包非常类似的语法设置中文字体:
\begin{command}
\cmd{setCJKmainfont}\oarg{font features}\marg{font name} \\
\cmd{setCJKsansfont}\oarg{font features}\marg{font name} \\
\cmd{setCJKmonofont}\oarg{font features}\marg{font name}
\end{command}
\cih{setCJKmainfont}
\cih{setCJKsansfont}
\cih{setCJKmonofont}

由于中文字体少有对应的粗体或斜体,\oarg{font features} 里多用其他字体来配置,
比如许多人习惯将宋体的 \texttt{BoldFont} 配置为黑体,而 \texttt{ItalicFont} 配置为楷体。

\section{段落格式和间距}

\subsection{行距}

如果你想在文档中使用更大的行距,你可在导言区使用如下命令进行设定:
\begin{command}
\cmd{linespread}\marg{factor}
\end{command}
\cih{linespread}

不过这个数字有一点点讲究:通常情况下,行与行之间有个基本间距,大小是字号的 1.2 倍,例如标准文档类使用
\texttt{10pt} 选项时,基本行间距是 12pt。而 \cmd{linespread} 是在 1.2 倍的基础上乘以 \Arg{factor}。
所以要想设定 1.5 倍行距的话,应当设 \cmd{linespread}\marg*{1.25} 。

如果 \cmd{linespread} 命令放在文档内部,它并不是及时生效的。你若是想要在某一个段落中局部地改变行距,
需要用到一个命令使 \cmd{linespread} 的改动立即生效
(本手册为使汉字和拉丁文字能够搭配,设定了1.5倍行距,我们用一个2倍行距的例子对比):
\begin{command}
\cmd{selectfont}
\end{command}

\begin{example}
{\linespread{1.67}\selectfont
This paragraph is typeset with
the baseline skip set to 2.0 times
the font size. Note the par
command at the end of the
paragraph.\par}

This paragraph has a clear
purpose, it shows that after the
curly brace has been closed,
everything is back to normal.
\end{example}

\subsection{段落格式}

以下命令分别设置段落的左缩进、右缩进和首行缩进:
\begin{command}
\cmd{setlength}\marg*{\cmd{leftskip}}\marg*{20pt}  \\
\cmd{setlength}\marg*{\cmd{rightskip}}\marg*{20pt} \\
\cmd{setlength}\marg*{\cmd{parindent}}\marg*{2em}
\end{command}

它们和设置字号的命令一样,在分段时生效。

\LaTeX\ 默认在段落开始时缩进,长度为你用上述命令设置的 \cmd{parindent}。如果你在某一段不想使用缩进,可使用某一段开头使用
\begin{command}
\cmd{noindent}
\end{command}
命令。相反地,
\begin{command}
\cmd{indent}
\end{command}
命令强制开启一段首行缩进的段落。

\LaTeX\ 还默认\textbf{在 \cmd{chapter}、\cmd{section} 等章节命令之后的第一段不缩进}。
如果你想使之缩进,当然可以使用 \cmd{indent} 逐个调整段落,但更简单的方式是在导言区使用 \pai{indentfirst} 宏包:
\begin{command}
\cmd{usepackage}\marg*{indentfirst}
\end{command}

段与段之间的垂直间距为 \cmd{parskip}:
\begin{command}
\cmd{setlength}\marg*{\cmd{parskip}}\marg*{1ex plus 0.5ex minus 0.2ex}
\end{command}

如上命令设置段落间距为弹性长度,可在 \texttt{0.8ex} 到 \texttt{1.5ex} 变动。

\subsection{水平间距}

\LaTeX 默认为单词之间增添了水平间距。我们可以用已经在数学公式中出现的 \cmd{quad} 和 \cmd{qquad} 命令制造一个额外的间距。
但是如果想要得到任意长度的间距,需要用到如下命令:
\begin{command}
\cmd{hspace}\marg{length}
\end{command}
\cih{hspace}

\cmd{hspace} 命令生成的间距如果位于一行的开头或末尾,则有可能被“吞掉”。这时可以使用 \cmd{hspace*} 代替 \cmd{hspace} 命令
得到不会因断行而消失的水平间距。

\begin{example}
This\hspace{1.5cm}is a space
of 1.5 cm.
\end{example}

命令 \cmd{stretch} 生成一个特殊弹性长度,用在 \cmd{hspace} 的参数里。它可以一直延伸,直到一行内剩余的空隙都被填满:
\begin{command}
\cmd{stretch}\marg{n}
\end{command}
如果同一行内出现多个 \cmd{hspace}\marg*{\cmd{stretch}\marg{n}},这一行的所有可用空间将以每个 \cmd{stretch} 设置的权重 \Arg{n} 进行分配。

\begin{example}
x\hspace{\stretch{1}}
x\hspace{\stretch{3}}x
\end{example}

在正文中用 \cmd{hspace} 调节水平间距时,往往使用 \texttt{em} 作为单位,它会随字号大小而变:

\begin{example}
{\Large{}big\hspace{1em}y}\\
{\tiny{}tin\hspace{1em}y}
\end{example}

\subsection{垂直间距}

在页面中,段落、章节标题、行间公式、列表、浮动体等元素之间的间距是 \LaTeX\ 预设的。比如 \cmd{parskip} ,默认设置为 \texttt{0pt plus 1pt}。

如果我们想要人为地增加段落之间的垂直间距,可以在\textbf{两个段落之间}的位置使用如下命令:
\begin{command}
\cmd{vspace}\marg{length}
\end{command}
\cih{vspace}

\cmd{vspace} 的间距在一页的顶端或底端可能被“吞掉”,类似 \cmd{hspace} 在一行的开头和末尾那样。
对应地,\cmd{vspace*} 命令产生不会因断页而消失的垂直间距。

在段落内部的某两行之间增加垂直间距,一般通过给 \cmd{\char`\\} 命令加上可选参数。这个办法也可以用于表格:
\begin{command}
\cmd{\char`\\}\oarg{length}
\end{command}

另外 \LaTeX\ 还提供了\cmd{bigskip}, \cmd{medskip}, \cmd{smallskip} 来增加预定义长度的垂直间距。

\section{页面尺寸}

我们不妨回顾一下最开头的文档类属性。\LaTeX\ 允许你通过文档类选项控制纸张的大小,包括 \texttt{a4paper}、\texttt{letterpaper}
(美国纸张标准,8.5in$\times$11in)等等,并配合字号设置了适合的页边距。这些页面参数由图 \ref{fig:layouts} 里给出的各种命令控制。

\begin{figure}[!p]
\currentpage
\oddpagelayouttrue
\pagediagram
\caption{控制页面的各种参数示意图} \label{fig:layouts}
\end{figure}

但是,如果你想要直接设置页边距等参数,着实是一件麻烦事。我们根据图 \ref{fig:layouts} 将各个方向的页边距计算公式给出(以奇数页为例):
\begin{equation*}
\begin{aligned}
\langle\text{\itshape left-margin}\rangle   &= \text{\ttfamily 1in} 
                                             + \text{\cmd{hoffset}}
                                             + \text{\cmd{oddsidemargin}} \\
\langle\text{\itshape right-margin}\rangle  &= \text{\cmd{paperwidth}} 
                                             - \langle\text{\itshape left-margin}\rangle
                                             - \text{\cmd{textwidth}} \\
\langle\text{\itshape top-margin}\rangle    &= \text{\ttfamily 1in} 
                                             + \text{\cmd{voffset}}
                                             + \text{\cmd{topmargin}}
                                             + \text{\cmd{headheight}}
                                             + \text{\cmd{headsep}} \\
\langle\text{\itshape bottom-margin}\rangle &= \text{\cmd{paperheight}}
                                             - \langle\text{\itshape top-margin}\rangle
                                             - \text{\cmd{textheight}}
\end{aligned}
\end{equation*}
如果我们想设置合适的 \Arg{left-margin} 和 \Arg{right-margin},就要靠上述方程组把 \cmd{oddsidemargin} 和 \cmd{textwidth} 等参数解出来!

幸好 \pkg{geometry} 宏包能够帮我们完成背后繁杂的计算,让我们能够用简便一些的方法设置页面参数。

\subsection{利用 \pkg{geometry} 宏包设置页面参数}

你既可以调用 \pkg{geometry} 宏包然后用其提供的 \cmd{geometry} 命令设置页面参数:
\begin{command}
\cmd{usepackage}\marg*{geometry} \\
\cmd{geometry}\marg{geometry-settings}
\end{command}
也可以将参数作为宏包的选项使用:
\begin{command}
\cmd{usepackage}\oarg{geometry-settings}\marg*{geometry}
\end{command}

其中 \Arg{geometry-settings} 多以 \Arg{key}=\Arg{value} 的形式组织。

比如,符合 Microsoft Word 习惯的页面设定是 A4 纸张,上下边距 1 英寸,左右边距 1.25 英寸,于是我们可以通过如下两种等效的方式之一设定页边距:
\begin{verbatim}
\usepackage[left=1.25in,right=1.25in,%
  top=1in,bottom=1in]{geometry}
\usepackage[hmargin=1.25in,vmargin=1in]{geometry}
\end{verbatim}

又比如,需要设定周围的边距一致为1.25英寸,可以用更简单的语法:
\begin{verbatim}
\usepackage[margin=1.25in]{geometry}
\end{verbatim}

对于书籍等双面文档,习惯上奇数页右边、偶数页左边留出较多的页边距,而书脊一侧的奇数页左边/偶数页右边页边距较少。我们可以这样设定:
\begin{verbatim}
\usepackage[inner=1in,outer=1.25in]{geometry}
\end{verbatim}

\pkg{geometry} 宏包本身也能够修改纸张大小、页眉页脚高度、边注宽度等等参数。更详细的用法在此不赘述,
感兴趣的用户可查阅 \pkg{geometry} 宏包的帮助文档。

\endinput